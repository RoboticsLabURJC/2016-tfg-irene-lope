\chapter{Práctica: Aspiradora autónoma (con autolocalización)}\label{cap.roomba}
En este capítulo se expondrá el desarrollo de una nueva práctica para la plataforma de JdeRobot-Academy, denominada ``Aspiradora autónoma (con autolocalización)''. Se aborda el desarrollo de su infraestructura, su componente académico correspondiente, así como el evaluador automático creado y la solución de referencia llevada a cabo. 

\section{Enunciado} \label{sec.enunciado}
En esta nueva práctica, el objetivo principal es lograr que un robot aspirador autónomo consiga barrer la mayor superficie posible de un apartamento. Para ello deberá hacer uso de su capacidad de autolocalización y del mapa de dicho apartamento. Esta aspiradora además de poseer un sensor de posición que le permite saber su posición y orientación, tiene un actuador de movimiento con el que controla su velocidad lineal y velocidad de giro y, un sensor láser que le permite medir la distancia a la que se sitúan los obstáculos (no será utilizado en el desarrollo de la solución de referencia). \\

En esta práctica el alumno deberá programar un algoritmo capaz de recorrer un gran porcentaje de la casa usando la capacidad de autolocalización del robot que deberá ofrecer la planificación de una ruta en forma de zig-zag. Después, deberá proporcionar un algoritmo de pilotaje para que la aspiradora siga dicha ruta. En la interfaz gráfica creada se puede observar el mapa de la casa, así como la posición de la aspiradora en dicho mapa y la superficie recorrida. Este mapa estará disponible para realizar el algoritmo de control y también se usará para la medida de la nota final.

\section{Infraestructura} 
En este apartado se describirá el entorno que se ha creado para poder realizar la práctica ``Aspiradora autónoma (con autolocalización)''. Primero se describirá el modelo del robot aspirador utilizado, incluyendo sus sensores y actuadores. Después, se explicará el mundo simulado (un apartamento típico, con distintas habitaciones y muebles) por el cual se moverá la aspiradora. 

\subsection{Modelo Roomba}
\subsection{Modelo house\_int2}
\subsection{Mundo de Gazebo}

\section{Componente académico}
\subsection{Interfaz gráfica}

\section{Solución de referencia} 

\section{Evaluador automático} 

\section{Experimentación} 

