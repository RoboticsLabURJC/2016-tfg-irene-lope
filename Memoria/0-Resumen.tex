\chapter*{Resumen}

La robótica es una rama de la ingeniería que emplea la informática para diseñar y desarrollar sistemas que permitan facilitar la vida del ser humano, e incluso sustituirle en determinadas tareas. Se pueden encontrar robots en diferentes áreas y entornos como la industria, la automovilística, la medicina o la domótica, entre otros. Debido a que actualmente la robótica está en continua expansión, es necesario formar a personas para que sean capaces de programar estos robots. \\

El objetivo de este proyecto es la creación de dos nuevas prácticas en el entorno de JdeRobot Academy. Estas prácticas están programadas en el lenguaje Python y se ha hecho uso del simulador Gazebo para realizar las distintas pruebas hasta conseguir un resultado final. También, se ha utilizado la librería OpenCV para el proceso de tratamiento digital de las imágenes y la herramienta PyQt5 para el desarrollo de la interfaz gráfica de cada una de las prácticas.\\

La primera práctica, llamada \textit{``Reconocimiento de la señal stop''}, trata de conseguir que un coche autónomo sea capaz de reconocer una señal de stop. Además, una vez detecte la señal, el coche tendrá que frenar en un cruce y detectar si hay otros coches circulando por la carretera. En el caso de que no aparezcan coches, deberá realizar un giro a la izquierda o a la derecha del cruce y continuar su camino. Para ello, se utilizarán tres cámaras de vídeo instaladas en el robot (una en el techo y dos en los faros del coche). \\

La segunda práctica, \textit{``Aspiradora autónoma con autolocalización''}, consiste en que una aspiradora autónoma barra la mayor parte de superficie posible de un apartamento. Dicha aspiradora posee el mapa de la casa y tiene instalado un sensor de movimiento que le permite saber en todo momento cual es su ubicación en el mundo.\\

Por último, se han propuesto unos trabajos futuros para poder mejorar y completar las prácticas desarrolladas en este proyecto.

