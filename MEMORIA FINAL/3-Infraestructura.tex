\chapter{Infraestructura}\label{cap.infraestructura}
En este capítulo se explicarán los principales ingredientes software en los que nos hemos apoyado para desarrollar el trabajo. Tales como el simulador Gazebo, el entorno JdeRobot, la biblioteca de OpenCV (de tratamiento de imagen), la biblioteca PyQt (para interfaces gráficas) y Python como lenguaje de programación.

\section{Simulador Gazebo}

Es un simulador usado en robótica que permite utilizar diversos escenarios tridimensionales virtuales donde probar nuestro software robótico. A la hora de desarrollar el software es necesario hacer pruebas, las cuales saldrían muy costosas si se probaran en robots reales (podría no funcionar correctamente y que el robot se rompiera). Por esta razón es muy útil el empleo de simuladores, pues se pueden realizar las pruebas que se quieran sin peligro de que el robot se estropee. Con los simuladores se pueden diseñar robots y escenarios realistas donde ejecutar los algoritmos creados.

\begin{figure}[H]
  \begin{center}
    \includegraphics[width=0.9\textwidth]{figures/Infraestructura/gazebo.png}
		\caption{Simulador Gazebo}
		\label{fig.gazebo}
		\end{center}
\end{figure}
 
Gazebo \footnote{\url{http://gazebosim.org/}} es un programa de código abierto distribuido bajo licencia de Apache 2.0. Se emplea en el desarrollo de aplicaciones robóticas y en inteligencia artificial. Fue elegido para realizar el DARPA Robotics Challenge (2012-2015) y está mantenido por la Fundación Open Robotics \footnote{\url{https://www.openrobotics.org/}}. Es capaz de simular robots, objetos y sensores en entornos complejos de interior y exterior. Tiene gráficos de gran calidad y un robusto motor de física (masa del robot, rozamiento, inercia, amortiguamiento, etc.).\\


El simulador Gazebo es uno de los ejes principales de JdeRobot-Academy. En este proyecto se emplea la versión 7 de Gazebo, la cual se usará para crear los diferentes entornos de las dos prácticas y para probar nuestras soluciones de referencia.\\

Gracias a Gazebo se pueden incluir texturas, luces y sombras en los escenarios, así como simular física como por ejemplo choques, empujes, gravedad, etc. Además, incluye diversos sensores, como pueden ser cámaras y lásers, los cuales podrán ser incorporados en los robots que empleemos. Todo ello hace que sea una herramienta muy potente y de gran ayuda en el mundo de la robótica.\\

Los mundos simulados con Gazebo son 3D, que se cargan a partir de ficheros con extensión ``.world''. Son ficheros definidos en SDF (Simulation Description Format) que es un formato XML (Extensible Markup Language) que describe objetos y entornos para simuladores, visualización y control de robots. Originalmente fue desarrollado como parte del simulador Gazebo pero con los años se ha convertido en un formato estándard estable, robusto y extensible capaz de describir todos los aspectos de los robots, los objetos estáticos y dinámicos, la iluminación, el terreno e incluso la física.\\

Se puede describir con precisión todos los aspectos de un robot usando SDF, ya sea un robot que sea un simple chasis con ruedas o un humanoide. Además de los atributos cinemáticos y dinámicos, se pueden definir sensores, propiedades de superficie, texturas, fricción de las juntas y muchas más propiedades de un robot. Estas características permiten usar SDF para simulación, visualización, planificación de movimiento y control de robot.\\

Los modelos de robots que se emplean en la simulación pueden ser creados mediante algún programa de modelado 3D como Blender o Sketchup. Estos robots simulados necesitan también ser dotados de inteligencia para lo cual se emplean \textit{plugins}. Éstos, pueden dotar al robot de comportamiento autónomo u ofrecer la información de sus sensores a aplicaciones externas y recibir de éstas comandos para los actuadores de los robots.


\section{Entorno JdeRobot}

JdeRobot \footnote{\url{http://jderobot.org/Main_Page}} es un middleware de software libre para el desarrollo de aplicaciones con robots y visión artificial. Esta plataforma fue creada por el Grupo de Robótica de la Universidad Rey Juan Carlos en 2003 y está licenciada como GPLv3 \footnote{\url{https://www.gnu.org/licenses/quick-guide-gplv3.html}}.\\

Está desarrollado en C y C++, aunque contiene componentes desarrollados en lenguajes como Python y Java. Proporciona un entorno de programación distribuido basado en componentes, donde el programa de aplicación se compone de una colección de varios componentes concurrentes asincrónicos. Dichos componentes operan entre sí mediante el middleware de comunicación ICE(Internet Communications Engine) o ROS messages. Los componentes pueden tener su propia GUI (Graphical User Interface).\\

La obtención de mediciones del sensor o el envío de órdenes al motor se realizan llamando a funciones locales. La plataforma conecta esas llamadas a componentes del driver que están conectados a dispositivos hardware del robot (reales o simulados, remotos o locales). Esas funciones conforman la API de la capa de abstracción de hardware. \\

Las aplicaciones constan de uno o varios componentes. Los que interactúan directamente con los sensores y actuadores del robot se llaman drivers, que son los encargados de controlar los robots y ofrecer interfaces de red con mensajes ICE o con ROS messages. Otros llevan en su código las funciones perceptivas, procesamiento de señales o la lógica de control e inteligencia del robot. En la Figura~\ref{fig.jderobot} se puede ver un ejemplo de esta comunicación con un AR Drone empleando interfaces ICE. La misma lógica de comportamiento se puede conectar al driver del drone real o al drive del drone simulado, basta con cambiar la configuración.

\begin{figure}[H]
  \begin{center}
    \includegraphics[width=0.9\textwidth]{figures/Infraestructura/jderobot.png}
		\caption{Ejemplo de componentes JdeRobot}
		\label{fig.jderobot}
		\end{center}
\end{figure}

Algunos de los dispositivos y robots actualmente soportados en JdeRobot son:
\begin{itemize}
	\item Sensores RGBD: Kinect y Kinect2 de Microsoft, Asus Xtion tanto reales como simulados en Gazebo.
	\item Robots de ruedas: TurtleBot de Yujin Robot y Pioneer de MobileRobotics Inc. tanto en real como en Gazebo.
	\item ArDrone quadrotor de Parrot tanto real como en Gazebo.
	\item Escáneres láser: LMS de SICK, URG de Hokuyo y RPLidar tanto reales como en Gazebo.
	\item Cámaras Firewire, cámaras USB, archivos de vídeo (mpeg, avi ...), cámaras IP (como Axis) tanto en real como en Gazebo.
\end{itemize}


JdeRobot también incluye varias herramientas y bibliotecas de programación de robots como:
\begin{itemize}
	\item Teleoperadores para varios robots, viendo sus sensores y controlando sus motores. 
	\item Herramienta ``VisualStates'' para programar el comportamiento del robot empleando autómatas
de estado finito.
	\item Herramienta ``Scratch2JdeRobot'' para programar robots (incluidos los drones) con el lenguaje gráfico de bloques estándar.
	\item Un calibrador de cámara.
	\item Una herramienta para ajustar filtros de color llamada ``ColorTuner''.
	\item Una biblioteca para desarrollar controladores borrosos y una biblioteca para la geometría proyectiva y el procesamiento de visión artificial.
\end{itemize}

En el desarrollo del proyecto se empleará la versión 5.6.2 de JdeRobot, ya que es la última versión estable. El nodo académico que se desarrollará para cada una de las prácticas es una aplicación JdeRobot.

\section{Lenguaje de programación Python}

Python \footnote{\url{https://www.python.org/}} es uno de los ejes principales de JdeRobot-Academy. Es el lenguaje de programación en el que están escritos los componentes académicos y las soluciones por lo que es el lenguaje utilizado para el desarrollo de este proyecto. Se trata de un lenguaje fácil de aprender y de alto nivel, es decir, se caracteriza por expresar los algoritmos de una manera adecuada a la capacidad cognitiva humana. Su creador fue Guido van Rossum, un investigador holandés que trabajaba en el centro de investigación CWI (Centrum Wiskunde \& Informatica). Actualmente Python es un lenguaje de código abierto administrado por Python Software Foundation. \\

Algunas funcionalidades incluidas en Python son la programación orientada a objetos, el manejo de excepciones, listas y diccionarios entre otras. A pesar de todo lo que soporta, se creó con el objetivo de que fuera un lenguaje sencillo de entender, sin perder todas las funcionalidades que pueden ofrecer lenguajes complejos tales como C. Incluye módulos que permiten la entrada y salida de ficheros, sockets, llamadas al sistema e incluso interfaces gráficas como Qt. Además, permite dividir el programa en módulos reutilizables y no es necesario compilarlo, pues es un lenguaje interpretado. \\

La última versión ofrecida por Python Software Foundation es la 3.6.2, pero en nuestro caso se empleará la 2.7.12 por compatibilidad con JdeRobot 5.5.2, que a su vez, sigue en esa versión de Python para ser compatible con ROS Kinetic.


\section{Biblioteca OpenCV}
OpenCV \footnote{\url{http://opencv.org/}} es una librería de código abierto desarrollada por Intel y publicada bajo licencia de BSD. Sus siglas provienen de los términos anglosajones ``Open Source Computer Vision Library''. Esta librería implementa gran variedad de herramientas para la interpretación de la imagen.\\

La biblioteca cuenta con más de 2500 algoritmos optimizados, que incluyen un conjunto completo de algoritmos de visión artificial y de aprendizaje automático, tanto clásicos como avanzados. Estos algoritmos se pueden usar para detectar y reconocer rostros, identificar objetos, rastrear movimientos de la cámara, rastrear objetos en movimiento, extraer modelos 3D de objetos, unir imágenes para producir una alta resolución imagen de una escena completa, encontrar imágenes similares de una base de datos de imágenes, etc. También tiene una librería de aprendizaje automático (MLL, Machine Learning Library) destinada al reconocimiento y agrupación de patrones estadísticos. \\

Fue diseñado para tener una alta eficiencia computacional. Está escrito en C/C++ y puede aprovechar las ventajas de los procesadores multinúcleo. Los algoritmos se basan en estructuras de datos flexibles acopladas con estructuras IPL (Intel Image Processing Library), aprovechándose de la arquitectura de Intel en la optimización de más de la mitad de las funciones. También se puede aprovechar del uso de tarjetas gráficas para acelerar el procesamiento.\\

Desde su aparición OpenCV ha sido usado en numerosas aplicaciones entre las cuales se encuentran la unión de imágenes de satélites y de mapas web, la reducción de ruido en imágenes médicas, los sistemas de detección de movimiento, la calibración de cámaras, el manejo de vehículos no tripulados o el reconocimiento de gestos. OpenCV es empleado también en reconocimiento de música y sonido, mediante la aplicación de técnicas de reconocimiento de visión en imágenes de espectrogramas del sonido. OpenCV ha sido usada en el sistema de visión del vehículo no tripulado Stanley de la Universidad de Stanford, el ganador en el año 2005 del Gran desafío DARPA. También hay una gran cantidad de empresas y centros de investigación que emplean estas técnicas como IBM, Microsoft, Intel, SONY, Siemens, Google, Stanford, MIT, CMU, Cambridge e INRIA.\\

Esta librería puede ser usada en distintos sistemas operativos como MAC, Windows, Android y Linux, y existen versiones para lenguajes de programación como C\#, Python y Java, a pesar de que originalmente era una librería en C/C++. Además, hay interfaces en desarrollo para Ruby, Matlab y otros lenguajes.\\

En este trabajo se ha empleado la versión 3.2 de OpenCV en Python. Esta librería se empleará para realizar todo lo relacionado con el tratamiento digital de imágenes en el ejercicio del coche autónomo. Con ello se extraerán datos que puedan emplearse a la hora de tomar decisiones para que los robots funcionen correctamente.

\section{Bliblioteca PyQt}
PyQt \footnote{\url{https://riverbankcomputing.com/software/pyqt/intro}} es un conjunto de enlaces Python para el conjunto de herramientas Qt, las cuales se emplean para el desarrollo de interfaces gráficas. Fue desarrollado por Riverbank Computing Ltd y es soportado por Windows, Linux, Mac OS/X, iOS y Android.\\

Qt es un entorno multiplataforma orientado a objetos desarrollado en C++  que permite desarrollar interfaces gráficas e incluye sockets, hilos, Unicode, bases de datos SQL, etc. PyQt combina todas las ventajas de Qt y Python, pues permite emplear todas las funcionalidades ofrecidas por Qt con un lenguaje de programación tan sencillo como Python. En concreto PyQt5 es un conjunto de enlaces Python para Qt5, disponible en Python 2.x y 3.x. Tiene más de 620 clases y 6000 funciones y métodos. PyQt5 dispone de una licencia dual, es decir, los desarrolladores pueden elegir entre una licencia GPL (General Public Licence) o una licencia comercial.\\ 

La interfaz gráfica de los componentes académicos creados en las prácticas de este TFG está escrita usando PyQt5. Las clases se dividen en ciertos módulos, tales como QtCore, QtGui, QtWidgets, QtXml o QtSql. En las prácticas desarrolladas se ha hecho uso de los siguientes módulos:
\begin{itemize}
	\item QtCore: contiene las funcionalidades principales que no tienen que ver con la interfaz gráfica. Este módulo se emplea para trabajar con archivos, diferentes tipos de datos, hilos, procesos, url, etc.
	\item QtGui: contiene clases para el desarrollo de ventanas, gráficos 2D, imágenes y texto.
	\item QtWidgets: dispone de clases que proporcionan un conjunto de elementos de interfaz de usuario para crear interfaces de usuario clásicas de escritorio. 
\end{itemize}

